\documentclass[11pt,a4paper]{article}
\usepackage[margin=2cm, top=3cm, includefoot]{geometry} 

\usepackage[utf8]{inputenc}
\usepackage[spanish]{babel}
\usepackage[parfill]{parskip}
\usepackage[hidelinks]{hyperref}
\usepackage{fancyvrb}
\usepackage{listings}
\usepackage{xcolor}
\usepackage{graphicx}
\usepackage{fancyhdr}

\definecolor{codegreen}{rgb}{0,0.6,0}
\definecolor{codegray}{rgb}{0.5,0.5,0.5}
\definecolor{codepurple}{rgb}{0.58,0,0.82}
\definecolor{backcolour}{rgb}{0.95,0.95,0.92}
\definecolor{background}{HTML}{dbdbdb}

\lstdefinestyle{mystyle}{
    backgroundcolor=\color{backcolour},   
    commentstyle=\color{codegreen},
    keywordstyle=\color{magenta},
    numberstyle=\tiny\color{codegray},
    stringstyle=\color{codepurple},
    basicstyle=\ttfamily\footnotesize,
    breakatwhitespace=false,         
    breaklines=true,                 
    captionpos=b,                    
    keepspaces=true,                 
    numbers=left,                    
    numbersep=5pt,                  
    showspaces=false,                
    showstringspaces=false,
    showtabs=false,                  
    tabsize=2
}

newcommand{gitversion}{ae26833}
newcommand{gitdate}{2024-07-12}
newcommand{gitcreated}{2024-07-12}


\lstset{style=mystyle}
\newcommand{\enterpriselogo}{images/colnex.png}

\addto\captionsspanish{\renewcommand{\contentsname}{Índice de Contenidos}}
\setlength{\headheight}{50pt}
\pagestyle{fancy}
\fancyhf{}
\rhead{\includegraphics[width=3cm]{\enterpriselogo}}
\renewcommand{\headrule}{\hbox to\headwidth{\color{background}\leaders\hrule height \headrulewidth\hfill}}

\begin{document}

\begin{titlepage}
    \centering
    \includegraphics[width=0.4\textwidth]{\enterpriselogo}
    \par
    \vfill
    {\LARGE\bfseries Walkthrough}
    \par\vspace{0.5cm}
    {\Large\bfseries Montaje del servicio de Front-End con Docker}
    \vfill
    \begin{flushright}
        {\large {\bfseries Fecha de Creación:} 2024/02/21}\par\vspace{0.2cm}
        {\large {\bfseries Última Actualización:} \gitdate{}}\par\vspace{0.3cm}
        {\large {\bfseries Versión del Documento:} \gitversion{}}
    \end{flushright}
  \vfill
  Este documento es confidencial.
  No debe ser impreso, divulgado o compartido con terceros.
\end{titlepage}

\pagenumbering{Roman}
\tableofcontents
\newpage
\pagenumbering{arabic}

\section{Descarga de la imagen de Docker}
El primer paso es adquirir la imagen desde la url en que se aloja:\par
\textit{https://drive.usercontent.google.com/download?id=1A\_FAfXEvndUtWB6G6i58qo3OV4jJt-Mq}\par
Para ello, podemos usar el comando \textit{curl} o la instrucción \textit{wget}
de la siguiente manera.

\begin{figure}[h]
    \begin{lstlisting}[language=Bash]
        // Comando para descargar la imagen con Wget
        wget --load-cookies /tmp/cookies.txt "url"

        // Comando para descargar la imagen con cURL
        curl -L -o frontend.tar.gz "url"
    \end{lstlisting}
\end{figure}

Adicionalmente, usted puede utilizar el recurso de la siguiente URL para descargar la imagen:\par
\centerline{\textit{\href{https://github.com/chentinghao/download_google_drive}{chentinghao/download\_google\_drive}}}

\vspace{5cm}

\section{Comprobación de la integridad del archivo}
Para verificar que el archivo que contiene la imagen de Docker se ha descargado de manera original 
y sin ninguna alteración, es una buena práctica contar con un hash que permita validar esto y comparar
que el que se genera con la copia descargada sea idéntico al original.\par

\begin{figure}[h]
    \begin{lstlisting}[language=Bash]
        // Hash original
        39BA72736A803089C15941B7C6C0249C
    \end{lstlisting}
\end{figure}

\begin{figure}[h]
    \begin{lstlisting}[language=Bash]
        // Comando para obtener el hash en Windows Powershell
        Get-FileHash frontend.tar.gz -Algorithm MD5

        // Comando para obtener el hash en Linux
        md5sum frontend.tar.gz
    \end{lstlisting}
\end{figure}

\newpage

\section{Montaje de la imagen de Docker}
Para instalar e iniciar la imagen Docker, es necesario ejecutar estos comandos:

\begin{figure}[h]
    \begin{lstlisting}[language=Bash]
        // Comando para importar la imagen de docker
        docker import frontend.tar.gz frontend:frontend

        // Comando para ejecutar la imagen de docker
        docker run -d -p 8090:80 --name "Frontend" -it frontend nginx -g "daemon off;"
    \end{lstlisting}
\end{figure}

\section{Notas de ejecución}
Normalmente el comando anteriormente utilizado para iniciar la imagen de Docker 
genera un error al no encontrar la imagen o contenedor \textit{frontend} por lo que, 
en su lugar, debemos seguir los siguientes pasos:\par

\subsection{Copiaremos el ID que aparece debajo del nombre de nuestra imagen en Docker Desktop}
\begin{figure}[h]
    \centerline{\includegraphics{images/img001.png}}
\end{figure}

\par
De no disponer de Docker Desktop, podremos obtener este identificador
al ejecutar el siguiente comando para posteriormente copiar el image ID:

\begin{figure}[h]
    \begin{lstlisting}[language=Bash]
        PS C:> docker images
        REPOSITORY   TAG        IMAGE ID       CREATED          SIZE
        frontend     frontend   148c955da4ca   55 minutes ago   195MB
    \end{lstlisting}
\end{figure}

\newpage

\subsection{Reemplazaremos la etiqueta frontend por el ID obtenido}

\begin{figure}[h]
    \begin{lstlisting}[language=Bash]
        // ejecutamos 148c955da4ca
        docker run -d -p 8090:80 --name "Frontend" -it 148c955da4ca nginx -g "daemon off;"
    \end{lstlisting}
\end{figure}

\end{document}