% parametros del documento
\documentclass[11pt,a4paper]{article}
\usepackage[margin=2cm, top=3cm, includefoot]{geometry} 

% importacion de modulos necesarios
\usepackage[utf8]{inputenc} % codificacion UTF-8
\usepackage[spanish]{babel} % para el idioma español
\usepackage[parfill]{parskip} % para ajustar el espaciado
\usepackage[hidelinks]{hyperref} % para la insercion de links
\usepackage{fancyvrb} % para el uso de formatos personalizados
\usepackage{listings} % para la insercion de código fuente
\usepackage{xcolor} % para ña insercion de colores

\usepackage{graphicx} % para la insercion de imagenes
\usepackage{fancyhdr} % para el ajuste de las cabeceras

% importacion de modulos externos 
newcommand{gitversion}{ae26833}
newcommand{gitdate}{2024-07-12}
newcommand{gitcreated}{2024-07-12}
 % para la obtencion de los datos de control

% definicion de variables de color
\definecolor{codegreen}{rgb}{0,0.6,0}
\definecolor{codegray}{rgb}{0.5,0.5,0.5}
\definecolor{codepurple}{rgb}{0.58,0,0.82}
\definecolor{backcolour}{rgb}{0.95,0.95,0.92}
\definecolor{background}{HTML}{dbdbdb}

% definicion de variables de imagenes
\newcommand{\enterpriselogo}{images/logo.png}

% definicion del formato de las inserciones de codigo
\lstdefinestyle{mystyle}{
    backgroundcolor=\color{backcolour},   
    commentstyle=\color{codegreen},
    keywordstyle=\color{magenta},
    numberstyle=\tiny\color{codegray},
    stringstyle=\color{codepurple},
    basicstyle=\ttfamily\footnotesize,
    breakatwhitespace=false,         
    breaklines=true,                 
    captionpos=b,                    
    keepspaces=true,                 
    numbers=left,                    
    numbersep=5pt,                  
    showspaces=false,                
    showstringspaces=false,
    showtabs=false,                  
    tabsize=2
}
\lstset{style=mystyle}

% cambio de nombre para la tabla de contenidos
\addto\captionsspanish{\renewcommand{\contentsname}{Índice de Contenidos}}

% definicion del formato de la cabecera
\setlength{\headheight}{50pt}
\pagestyle{fancy}
\fancyhf{}
\rhead{\includegraphics[width=3cm]{\enterpriselogo}}
\renewcommand{\headrule}{\hbox to\headwidth{\color{background}\leaders\hrule height \headrulewidth\hfill}}

% inicio del documento
\begin{document}

% portada
\begin{titlepage}
    \centering
    \includegraphics[width=0.4\textwidth]{\enterpriselogo}
    \par
    \vfill
		{\LARGE\bfseries Reporte confidencial XXXXXXXXXX} % titulo principal del documento
    \par\vspace{0.5cm}
    {\Large\bfseries Documento Técnico} % titulo secundario del documento
    \vfill
		% control de versiones
    \begin{flushright}
        {\large {\bfseries Fecha de Creación:} \gitcreated}\par\vspace{0.2cm}
        {\large {\bfseries Última Actualización:} \gitdate{}}\par\vspace{0.3cm}
        {\large {\bfseries Versión del Documento:} \gitversion{}}
    \end{flushright}
  \vfill
  Este documento es confidencial.
  No debe ser impreso, divulgado o compartido con terceros.
\end{titlepage}

% tabla de contenidos
\pagenumbering{Roman}
\tableofcontents
\newpage
\pagenumbering{arabic}

% inicio de contenidos

\section{Introducción}
Este documento pretende explicar de manera detallada...

\section{Objetivo}

\section{Alcance}

\section{Términos y definiciones}

\section{Desarrollo}

\section{Anexos}

% final de contenidos y del documento
\end{document}
